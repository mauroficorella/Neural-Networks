\documentclass[12pt]{article}
\usepackage[a4paper,
left=20mm,
right=20mm,
top=20mm,
bottom=20mm]{geometry}
\usepackage{amsmath}
\usepackage{graphicx}
\usepackage{algorithm}
\usepackage{algorithmic}
\usepackage{multicol}
\usepackage{amsthm}
\usepackage{bm}
\usepackage{fancyhdr}
\usepackage{amssymb}

\newcommand\tab[1][1cm]{\hspace*{#1}}
\renewcommand{\labelitemii}{$\star$}

\begin{document}


\begin{titlepage}
	\begin{center}
		\vspace*{1cm}
		
		\Huge
		\textbf{FastGAN: Faster and Stabilized GAN}
		\vspace{1.5cm}
		
		\Large
		Authors:\\
		\textbf{Mauro Ficorella 1941639}\\
		\textbf{Martina Turbessi 1944497}\\
		\textbf{Valentina Sisti 1952657}\\
		\vspace{0.5cm}
		
		\vfill
		
		\includegraphics[width=0.4\textwidth]{Images/Logo.jpg}
		
		\vfill
		
		\vspace{0.8cm}
		
		\Large
		Sapienza\\
		May 2021
	\end{center}
\end{titlepage}


\newpage
\pagestyle{fancy}
\fancyhf{}
\rhead{Mauro Ficorella, Martina Turbessi, Valentina Sisti}
\lhead{FastGAN}

% ABSTRACT --------------------------------------------------------------------

%\section*{Abstract}
\begin{center}
	\normalsize\MakeUppercase{\textbf{Abstract}}

	\begin{minipage}[t]{0.8\textwidth}
	\textit{The main aim of FastGAN is to allow users with limited computing budget and resources to 
	train a GAN. Moreover it eliminates the requirement of a big dataset for training.
	These are big advantages since traditional GANs required a lot of GPU computational power
	(i.e. one or more server-level GPUs with at least 16 GB of vRAM in StyleGAN2) and a large number of 
	images for training. 
	This implementation allowed to train from scratch on a NVIDIA GeForce RTX 2070 SUPER and a 
	NVIDIA GeForce GTX 1050-Ti, obtaining good results also on a small dataset. 
	The structure of FastGAN comprehends a Skip-Layer channel-wise Excitation (SLE) module and a self-supervised
	Discriminator trained as a feature-encoder.\\
	We will show, through various experiments, that this GAN outperforms StyleGAN2 in terms of computational requirements
	and training time.
	}
	\end{minipage}

\end{center}


% INTRODUCTION --------------------------------------------------------------------

\section{Introduction}
	

% CAPITOLO 1 -------------------------------------------------------------------------

\section{Method}
	


% CAPITOLO 2 -------------------------------------------------------------------------

\newpage
\pagestyle{fancy}
\fancyhf{}
\rhead{NOMI O CAPITOLO}
\lhead{TITOLO PROGETTO}

\section*{NOME}

% CAPITOLO 3 -------------------------------------------------------------------------

\newpage
\pagestyle{fancy}
\fancyhf{}
\rhead{NOMI O CAPITOLO}
\lhead{TITOLO PROGETTO}

\section*{NOME}

% CAPITOLO 4 -------------------------------------------------------------------------

\newpage
\pagestyle{fancy}
\fancyhf{}
\rhead{NOMI O CAPITOLO}
\lhead{TITOLO PROGETTO}

\section*{NOME}

% CAPITOLO 5 -------------------------------------------------------------------------

\newpage
\pagestyle{fancy}
\fancyhf{}
\rhead{NOMI O CAPITOLO}
\lhead{TITOLO PROGETTO}

\section*{NOME}

% CAPITOLO 6 -------------------------------------------------------------------------

\newpage
\pagestyle{fancy}
\fancyhf{}
\rhead{NOMI O CAPITOLO}
\lhead{TITOLO PROGETTO}

\section*{NOME}

% CAPITOLO 7 -------------------------------------------------------------------------

\newpage
\pagestyle{fancy}
\fancyhf{}
\rhead{NOMI O CAPITOLO}
\lhead{TITOLO PROGETTO}

\section*{NOME}

% CAPITOLO 8 -------------------------------------------------------------------------

\newpage
\pagestyle{fancy}
\fancyhf{}
\rhead{NOMI O CAPITOLO}
\lhead{TITOLO PROGETTO}

\section*{NOME}

\end{document}

% COSE UTILI --------------------------------------------------------------------------

%\section*{NOME}
%\subsection*{1.1}
%\setlength{\intextsep}{0pt} --> elimina lo spazio
%\vspace{-3mm}
%\hspace*{0cm}

% Font -------------------------------------

%GRASSETTO: \textbf

% Simboli ---------------------------------

%$\leftarrow$

% Elenco puntato ----------------------

%\begin{itemize}
%\setlength\itemsep{0.01em}
%\item 1
%\item 2
%\end{itemize}

% Graffa grande -----------------------

%\[  
%    \left\{ 
%    \begin{array}{ll} 
%      \mbox{1}
%      \mbox{2}
%    \end{array}
%    \mbox{riga al lato}
%   \right. 
%\]

% Multicolonne --------------------------

% \begin{multicols}{2}
% \columnbreak
% \end{multicols}

% Algoritmi -------------------------------

%\renewcommand{\thealgorithm}{1.\arabic{algorithm}}
%\setcounter{algorithm}{0}
%\begin{algorithm}
%\footnotesize
%\caption{Nome}
%\textbf{Input:} \\
%\textbf{Output:} 

%\begin{algorithmic}[1]
%\STATE 
%\FOR{ = 0 \TO i = n} ---- \ENDFOR
%\IF{} ---- \ELSIF{} ---- \ENDIF
%\RETURN 
%\end{algorithmic}
%\end{algorithm}

% Minipage ------------------------------

%\begin{minipage}[t]{0.5\textwidth}

% queste 3 righe vanno attaccate
%\end{minipage}
%\hspace{0.02\linewidth}
%\begin{minipage}[t]{0.47\textwidth} 

%\begin{minipage}[t]{0.3\textwidth} 
%\end{minipage}

% Proof --------------------------------
%\begin{proof}[\textbf{per cambiare nome}]
%\end{proof}

